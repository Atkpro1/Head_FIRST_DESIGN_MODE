\documentclass{ctexart}
\def\enter{{\hfill\break}}
\begin{document}
不编鬼故事了.\enter
COMMAND\ PATTERN:代码变数据\enter
OMG,LISP的棺材压不住了!!!\enter
这里没啥好说的,要是我直觉上就会这么做.\enter
倒是后面的"更多用途"说明了它"代码变数据"的本质\enter
\enter
后面的ADAPTER和FACADE更有用些.\enter
ADAPTER实质上是某种情况下的必需品.而FACADE...是一种...神奇的...简化方案.\enter
\enter
偷偷告诉你,C\&C++中小FACADE由MACRO实现.\enter
这个MACRO...就是编译前来一次SUBSITUATION.\enter
FOR\ EXAMPLE:\enter
{\#define MOVIE\_START do\{Popper.on();/*...*/\}while(0)}\enter
\enter
提到类适配器,C++有一种private继承\enter
private继承让父类public和protected的Method和Property变为private,表达HAS\ ONLY\ ONE  (不是IS\ A)的关系\enter
哦,对了,还有protected继承,父类的东西于是就不是private而是protected的了.\enter
这两种继承关系似乎正适用于类适配器???\enter
现在没了鬼故事,没了抄书,只剩下补充的东东,篇幅短了呢!\enter
不过就这样吧,886(预计还有加餐)\enter
加餐来也!!!\enter
EMMM,我还能说啥呢???\enter
瞎说吧.\enter
ADAPTER你也看到了,他适用于新旧连接处.\enter
所以说,在你自己的代码里,它的出现表示你发现了一些类似的接口.\enter
尽可能统一接口吧.\enter
当然,如果这两个接口无法统一(使用同样的接口),那么就要ADAPTER上了.\enter
FACADE你会连你用了都不知道.\enter
例如--你写了一个类,传入一条SQL,返回一个SQL\_Lines\enter
这就是一个外观模式\enter
但外观模式仅仅是一堆常用代码的汇总,Java却要用一个"函数调用"来完成,C\&C++能不能省掉来加速呢??? \enter
能!你可以用宏或内联函数.\enter
宏其实就是替换,还是比较简单的\enter
提到宏,讲一下C\&C++的预处理.\enter
C\&C++程序编译前会被一个叫$CP_{re}P_{roccesser}$的程序处理一遍.\enter
这个玩意儿只处理"\#"开头的行,其他的从不解析,所以你也可以用他处理其他语言的代码.\enter
但他会生成另一些"\#"开头的行,所以你还得把这些"\#"去了.\enter

你可以看一看code/6A7/Before.cpp,然后用gcc~-E~Before.cpp看一看预处理的结果\enter 
虽然宏解决了一个函数调用的浪费,但它也有问题--\enter
宏的另一种形式--宏调用允许把一些参数传递给宏,但...\enter
Bad.cpp中,虽然()保证了没有结合错误,但++仍让它束手无策,.\enter
SO!\enter
C++提供了inline关键字,尽可能(不是一定!)节省这类函数调用\enter
但宏依旧存在.以它能在任何地方(包括函数定义在内)改变的神奇能力出现.\enter
当然,还有向后兼容233\enter
嗯,瞎扯扯够200字了(话说如果大家认为我该抄书的话尽管开口)\enter

\end{document}
