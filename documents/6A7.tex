\documentclass{ctexart}
\def\enter{{\hfill\break}}
\begin{document}
不编鬼故事了.\enter
COMMAND\ PATTERN:代码变数据\enter
OMG,LISP的棺材压不住了!!!\enter
这里没啥好说的,要是我直觉上就会这么做.\enter
倒是后面的"更多用途"说明了它"代码变数据"的本质\enter
\enter
后面的ADAPTER和FACADE更有用些.\enter
ADAPTER实质上是某种情况下的必需品.而FACADE...是一种...神奇的...简化方案.\enter
\enter
偷偷告诉你,C\&C++中小FACADE由MACRO实现.\enter
这个MACRO...就是编译前来一次SUBSITUATION.\enter
FOR\ EXAMPLE:\enter
{\#define MOVIE\_START do\{Popper.on();/*...*/\}while(0)}\enter
\enter
提到类适配器,C++有一种private继承\enter
private继承让父类public和protected的Method和Property变为private,表达HAS\ ONLY\ ONE  (不是IS\ A)的关系\enter
哦,对了,还有protected继承,父类的东西于是就不是private而是protected的了.\enter
这两种继承关系似乎正适用于类适配器???\enter
现在没了鬼故事,没了抄书,只剩下补充的东东,篇幅短了呢!\enter
不过就这样吧,886(预计还有加餐)\enter
\end{document}
