\documentclass{article}
\usepackage{ctex}

\def\enter{{\hfill\break}}

\begin{document}
大家好,我会来啦!

好了,现谈这正事,我们这次要阅读《HEAD FIRST设计模式》,又称《面向对象程序设计套路入门》

[对,套路,深深的套路]

不过这些套路一向很有用呢.

好了,这不是篇0吗,我们就别写引言了,我们来写点其他的事

这里的文章都发表在GITHUB,目录结构如下有4个文件夹

|主目录 \enter
|-------effect \enter
|       |一些用TEX写的效果,增加滑稽与可读性 \enter
|       |空的,别想啦 \enter
| \enter
|-------document \enter
|       |直接用TEX写的原文件,可供阅读 \enter
|       |0.tex 本文件 \enter
|	|X.tex 第X章原文件 \enter
| \enter
|-------code \enter
|       |用C++编写的代码,把中文换掉就可编译 \enter
|       |-------X \enter
|       |       |第X章的代码 \enter
|       |       |basic 第一版代码 \enter
|       |       |Change\_*** 用***方式修改后 \enter
|       |       |Pattern\_Used 使用了设计模式后 \enter
|       |       |Pattern\_Change\_*** 使用了***与设计模式修改后(有时不存在) \enter
|       |       |Pattern\_Result 结果方案 \enter
| \enter
|-------pdf \enter
|       |从document用xelatex程序编译的结果 \enter
|       |X.pdf X.tex的结果 \enter

什么,无法预览DVI?我不想提供PDF(懒),直接看X.tex也没什么问题.

(其实C++OOP坑人之处很多,例如构造函数不能调用抽象函数,不会自动继承构造函数......)

好了,大家可以去看第一章了,See You.

\end{document}
