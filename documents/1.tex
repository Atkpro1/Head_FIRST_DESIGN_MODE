\documentclass{ctexart}
\def\enter{{\hfill\break}}

\begin{document}
\enter
大家好啊.\enter
今天我们将遇到我们的第一个套路--策略模式.\enter
\enter
[产品乱入]:小许啊,我们这里有个需求文档,看看吧.
\enter
XX市管理局:\enter
贵公司好,我市希望请贵公司为我市设计一个管理系统,统计市民的生活情况.\enter
现在先请贵公司设计一个类,模拟一个市民.此市民可以吃饭,睡觉,工作.\enter
注:不同市民的生活方式可以不一样,我们会提供不同市民的动作.\enter
注2:具体行为的变化可能较快,希望贵公司谅解.\enter
注3:我们会尽可能快的发来下一篇需求文档.\enter
\enter
谢谢\enter
\enter
哎呀,今天晚上我吃臭鸡蛋和瓜皮吧(编的真烂)\enter
\enter
算了,先分析需求.\enter
\enter
需求不就是让我们写一个市民类吗,里面有eat,sleep,work3个方法,老简单了.\enter
小许很快写出了第一版代码basic.cpp\enter
\enter
这时,小许突然收到了一封邮件.\enter
\enter
PM:客户刚刚说不同市民的行为可能不一样.\enter
\enter
小许:切,我把函数改称支持多态的不就好了(PS.C++函数默认带final)\enter
很快,小许就写好了第二版的代码(Change\_Virtual.cpp),交给了BILL审批\enter
BILL:嘿嘿嘿,你和张大胖犯了同一个错误.去问他吧.\enter
\enter
小许:请问张大胖在吗?\enter
张大胖:在.\enter
小许:BILL说我这版代码不合格,为什么?\enter
张大胖满脸自豪(BILL优秀小学教师实锤):你知道设计模式吗?\enter
小许:布吉岛\enter
张大胖:就是一群前人总结的经验而已\enter
小许:那我去背不就是啦?\enter
张大胖:NO,NO,NO,你要理解其精华.\enter
小许:那其精华是什么呢???\enter
张大胖:你最讨厌什么?\enter
小许:RubbishPM改需求!\enter
张大胖:那你就要让改需求时更方便.\enter
小许:什么意思?\enter
张大胖:就是要让容易改变的地方做的容易改变,容易复用.\enter
小许:我这里只要一个类就可以了呀?\enter
张大胖:我问你,如果有100个吃饭,睡觉方法互不相同的包子师傅,你要用几个类?\enter
小许:100个.\enter
张大胖:产品要改一下包子的制作方法.\enter
小许:emmmm,100个类要改死我啊\enter
张大胖:你看,BILL就把你的代码退回来了吧.\enter
小许:哦!谢谢张大胖.\enter
张大胖:不谢(BILL还送我个小黄鸭,不错)\enter
\enter
小许想了想,决定增加一个SKILL类(源代码我其实直接写成了FunctionPointer)\enter
然后增加了SKILL类成员Eat,Sleep,Work.在eat,sleep,work里调用.(Pattern\_used.cpp)\enter
BILL看了一眼:不错,不过你还是把SKILL改成类吧,毕竟这个SKILL里可能会发生一些神奇的事情.改完就能Commit了\enter
很快,小许写出了最终版代码(Pattern\_Result.cpp)提交了.\enter
BILL:小许啊,做的不错,你知道这是什么模式吗?\enter
小许:什么模式?\enter
BILL:这叫做--策略模式,那个SKILL类就是"策略"\enter
小许:哦,涨芝士了.\enter
这时,小许的邮箱又响了......未完待续.\enter
\end{document}
