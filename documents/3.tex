\enter
小许:WHAT~XX~PM,NEW~REQUEST\enter
翻译君:XX产品,又加需求!\enter
贵公司好,我们希望能在市民吃饭,工作时增加一些对物品的需要,工作时能生产出一些物品.\enter
注1:这些物品分种类\enter
注2:这些物品可以由政府采购\enter
\enter
小许:什么鬼意思.\enter
BILL(记得吗,上期BILL路过):就是市民工作时要能搞点东西出来,吃饭睡觉时要用到这些东西.\enter
小许:哦\enter
BILL:我先去休息啦.你先写着.\enter
小许一段苦思冥想,写出了头一版代码.\enter
BILL:小许啊,我没时间和你整理,张大胖最近在学习设计模式,他应该可以帮到你.\enter
\centerline{-----------------------------}\enter
张(大胖我就不写了.):来,我们分两部分讲.先说SKILL类.\enter
小许:这怎么了?\enter
张:你知道吗,设计模式里有个原则,叫"一个类,一个责任"\enter
小许:可以理解.\enter
张:那你觉得能否把"资源"放到另一个类中去执行呢?\enter
小许:EMMM,这题我要看答案.\enter
张:嘿嘿,我一开始也没想出来,你看这段代码(Decorater.java)(张大胖不会C++)(我写JAVA可能会语法错)\enter
小许:额,SKILL的子类套SKILL,这不是俄罗斯套娃吗?\enter
张:是啊,套一层娃就是多了一个需要,这就把需要移到别的类里了,神奇BUT强大.\enter
小许:厉害了.\enter
张:你先去写,我喝口水,一会儿我们继续讲.\enter
